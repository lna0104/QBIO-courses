\documentclass[11pt]{article}

\usepackage[left=5em, right=5em, top=5em]{geometry}
\usepackage{cite}
\usepackage{amsmath}
\usepackage{xcolor}
\usepackage{enumitem}
\usepackage{url}
\PassOptionsToPackage{hyphens}{url}
\usepackage{hyperref}
\usepackage{amsfonts}
\usepackage{cite}

%\hypersetup{
%    colorlinks=true,
%    linkcolor=black,
%    citecolor=red,
%    filecolor=black,
%    urlcolor=blue,
%}

\begin{document}

\onecolumn
\begin{center}
\vspace*{8em}
{\huge {Sampling Design and Analysis in Conservation Science}}\\
\bigskip
\bigskip
{\huge Critical Essay}\\
\bigskip
\bigskip
\bigskip
\bigskip


{\large Le Na Ngo - 47615427}

\bigskip
\end{center}

\clearpage


\bigskip
\begin{center}

\textbf{Question 1:}

\end{center}

You have received a request from an agency responsible for managing tiger populations in India, “We are spending several million dollars annually on patrols to look for tiger poachers and illegal logging activities that are impacting tiger habitat. My boss wants to know if our patrols are effective in protecting tiger habitat and maintaining tiger populations. Can you propose a research programme to tell me this? I need your research programme to report within three years”. 

\bigskip
\bigskip
\begin{center}

\textbf{Solution:} 

\end{center}

The illegal killing and poaching of wildlife have brought numerous species to the edge of extinction. In the last decade, governments have allocated millions of dollars to fund patrols with the goal of preventing further declines in tiger populations in India. Consequently, it is crucial to assess the efficacy of these efforts and make necessary adaptations. The study raises the question of whether these patrols indeed contribute to the protection of tiger habitats and the preservation of tiger populations. 

Within this study, we evaluate the influence of forest ranger patrols on the presence of snare traps intended for tigers, the tiger population, the number of arrests and the subsequent prosecution rates.

The study was conducted within the Central Terai landscape of northern India, where ten patrol teams were formed, each comprising varying numbers of members. Each team was assigned responsibility for a separate area. For the purposes of this research, we operated under the assumption that all these areas share a similar habitat (e.g., forest cover). 

Three primary categories of data were employed in this study: i) the geographic coordinates and quantity of active snare traps set for tigers and their prey, ii) the recorded tiger population, and iii) the number of arrests related to tiger poaching in the study area. Each patrol team employed GPS devices and mapping tools to gather information on the location of snare traps. Concurrently, we assessed patrol efforts through various metrics, including the distance covered during patrols, the frequency of patrol trips, and the team's composition in terms of the number of members. For each year of the study, we determined the total number of snare traps set for tigers and their prey. From these figures, we calculated the catch-per-unit effort of the ranger teams by dividing it by the patrol effort~\cite{RISDIANTO2016306}. Each year, data regarding the tiger population and the number of arrests were recorded. This research spanned three years. These datasets were subjected to statistical analysis to evaluate their significance and examine potential correlations.

In the data analysis phase, our approach will commence with organizing and cleansing of raw data, ensuring its accuracy and comprehensiveness. After data preparation, we will employ descriptive statistics to offer a high-level overview of key trends and patterns spanning a three-year study period. Subsequently, we will generate multiple bar charts to visually depict discernible trends or disparities within each category of data across the three-year study duration. This graphical representation will facilitate the identification of potential outliers or irregularities. To investigate the relationships between the number of patrols conducted by each team and each type of collected data including the number of active snare traps, tiger population, and the number of arrests, we will conduct fundamental correlation analyses to determine the presence of any significant associations. Furthermore, we intend to employ the tiger population and the number of arrests recorded prior to the initiation of the patrol project as control variables. This will enable us to examine whether any significant differences emerge in these data when comparing periods with and without patrols. 

To conclude, we can answer the question mentioned at the beginning whether these patrols affect the protection of tiger habitats and the preservation of tiger populations. For example, our descriptive statistics reveal a noticeable decrease in the number of active snare traps for tigers across the three years of study, coinciding with the implementation of ranger patrols. This suggests a potential positive impact of the patrols on mitigating the threat of snare traps to tiger populations. Secondly, our correlation analyses indicate a strong inverse relationship between the frequency of ranger patrols and the number of tiger snare traps found. This statistically significant correlation underscores the effectiveness of patrols in reducing the presence of these traps, further supporting their role in tiger conservation efforts.  Our analysis of tiger population trends also reveals a gradual increase over the study duration. Although we cannot definitively attribute this growth solely to the presence of patrols, it is plausible that their existence contributes to the overall stability and potential expansion of the tiger population. Moreover, our observations also highlight a substantial decrease in the number of arrests associated with tiger poaching along with the increased number of patrols. It suggests that the patrols have an impact on combating illegal wildlife trade. Statistical tests of the tiger population and the number of arrests between control and patrols show significant p-values. This helps us to confidently reject the null hypothesis that patrols have no impact on tiger conservation. Nonetheless, the scenario described above represents just one potential outcome. If these patrols do not significantly impact the protection of tiger habitats and the preservation of tiger populations, we recommend that the government consider redirecting its funding towards alternative strategies. This assessment serves as a stepping stone for future conservation actions.


\bibliographystyle{unsrt}
\bibliography{ref}
\end{document}


